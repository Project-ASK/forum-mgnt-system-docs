\documentclass[twocolumn,10pt]{article}
\oddsidemargin 0.1in
\addtolength{\oddsidemargin}{-.4in}
\addtolength{\textwidth}{1in}
\addtolength{\textheight}{2in}
\addtolength{\headheight}{-1in}
\title{College of Engineering Chengannur
\\Department of Computer  Engineering
\\B. Tech. Computer Science \& Engineering 
\\CSD415 Project Phase I (2023)
\\ Project  Proposal \& Approval
\\ \bf{Web Application}
}

\author{Group No.1 Forum Management System }
\usepackage{epsfig}
\begin{document}

\maketitle

{\bf Keywords:} Next JS, Node JS, MERN, Javascript, Browser,
forum, information system, report
management system \\

\abstract{}
The project proposal encompasses the creation of a comprehensive website designed to effectively manage and monitor various activities within the forums and communities at CEC, including IEEE, IEDC, FOCES Tinkerhub, GDSC, TPC, and others. This website not only serves as a centralized hub for these entities but also provides valuable functionality for college staff to oversee forum activities, facilitating accreditation and audit processes. Beyond internal management, the site offers external visibility, enabling visitors to view past and upcoming events. By fostering streamlined management, transparency, and accessibility, the proposed website is poised to significantly enhance the coordination and visibility of college forums, ultimately contributing to a more organized and engaged community.\section{\label{intro}Introduction}
In response to the evolving needs of our academic community at College of Engineering Chengannur (CEC), we propose the development of a robust web application, the 'Forum Management System.' This innovative system aims to revolutionize the way we manage and oversee various student forums and communities, including IEEE, IEDC, Tinkerhub, GDSC, TPC, and others. The Forum Management System seeks to streamline administrative tasks, enhance transparency, and promote community engagement by providing a centralized hub for forum activities. In this proposal, we present a clear problem statement, innovative solutions, required infrastructure, and a tentative work schedule, all of which contribute to our mission of creating a more organized, efficient, and connected forum ecosystem at CEC.
\section{\label{work}Report of Work Done}
During the initial phases of our project, our team conducted extensive research and deliberation to identify the most suitable project idea, ultimately choosing to develop a Forum Management System. To ensure the success of this endeavor, we engaged in various key activities. First, we conducted thorough research into existing forum management systems, studying their functionalities, and analyzing their strengths and weaknesses. To better understand the unique needs and challenges faced by college forums and communities, we actively reached out to various forums within our institution. These interactions included discussions and interviews with forum members and administrators to gather valuable insights. In terms of technology selection, we explored a range of frontend and backend frameworks, as well as database management systems. We are currently in the process of finalizing our technology choices based on our project's specific requirements. Furthermore, we compiled a comprehensive list of features and functionalities that the web application will incorporate. This list serves as a blueprint for our development process, ensuring that we meet the specific needs of our users and stakeholders. These initial activities have set a strong foundation for our project, providing clarity on technology choices and project requirements, and positioning us for the successful development of the Forum Management System.
\section{Proposed Project}
The Forum Management System project aims to address a pressing issue at the College of Engineering Chengannur (CEC) by creating a comprehensive web-based solution for efficiently managing and coordinating various forums and communities. Currently, multiple forums, including IEEE, IEDC, Tinkerhub, GDSC, TPC, and others, operate independently without a centralized management system. This fragmentation leads to challenges in coordination, administrative overhead, limited visibility, and engagement. In response, our project endeavors to develop a robust web application that streamlines forum management, enhances transparency, and improves the overall forum experience for both coordinators and members. In this section, we provide a detailed breakdown of our proposed solution and the key features it will offer.
\subsection{\label{ps}Problem Statement}
Our project aims to develop a web application that centralizes forum management, ensures transparency, and enhances the experience for both coordinators and members, prioritizing precise database management and event coordination.
\subsection{Proposed Solution}
To address the identified challenges, we propose the development of the Forum Management System (FMS), a web-based application designed to serve as a centralized hub for all forums and communities at CEC. FMS will offer the following key features and functionalities:

\begin{itemize}
    

\item Centralized Management: FMS will provide a unified platform for forum coordinators and members to manage and coordinate events, activities, and communications. It will eliminate the need for disparate tools and spreadsheets.

\item Event Scheduling: Coordinators can easily schedule and manage forum events, including workshops, seminars, and meetings, through an intuitive event calendar.

\item Attendance Tracking: The system will automate attendance tracking, making it easier for coordinators to monitor member participation and engagement.

\item Communication Hub: FMS will facilitate seamless communication among forum members and coordinators through announcements, forums, and messaging features.

\item Transparency and Visibility: College staff and external stakeholders will have access to view forum activities, past events, and upcoming schedules, aiding in accreditation and audit processes.

\item Member Engagement: Forum members will receive notifications and updates about events, ensuring better participation and engagement.

\end{itemize}
\subsection{Hardware \& Software Requirements}
The implementation of the Forum Management System will require the following hardware and software resources:


\textbf{Hardware:}
\begin{itemize}
\item Personal laptop or computer (for development and testing purposes)
\item Internet connectivity
\end{itemize}

\textbf{Software:}

\begin{itemize}
\item Development tools such as Visual Studio Code (VSCode) for coding and project management.
\item Frontend development framework such as React \& NextJS
\item Backend development framework (e.g., Django, Express, Ruby on Rails) for web application development.
\item Database management system (e.g., MySQL, MongoDB, PostgreSQL) for data storage and retrieval.
\item Web hosting platform provided through the GitHub Student Pack.
\end{itemize}
\subsection{Work Schedule}
The project will be executed in a phased approach, ensuring a systematic and organized development process. The tentative work schedule for the project is as follows:

\begin{itemize}
  \item \textbf{August - September 2023:} In this initial phase, our focus will be on project planning, team formation, and assessing the availability of necessary resources. We will ensure that all required hardware and software resources are in place, and any deficiencies will be addressed promptly. 

  \item \textbf{October - November 2023:} These months will be dedicated to system design and architecture development. Our team will collaboratively design the structure and core functionality of the web application, outlining its key components and user interfaces.

  \item \textbf{December 2023 - March 2024:} This period marks the heart of our project, where we will engage in the extensive development of both the front-end and back-end components of the web application. The front-end will focus on creating a user-friendly and visually appealing interface, while the back-end will handle data processing and management.

  \item \textbf{April 2024:} Quality assurance and testing will be the primary activities during this month. We will rigorously test the web application to identify and rectify any bugs or issues. Our goal is to ensure a reliable and error-free user experience.

  \item \textbf{May 2024:} The final phase of the project will involve conducting comprehensive testing, refining the application based on user feedback, and ultimately completing the project. We will also prepare any necessary documentation and training materials.

\end{itemize}

This work schedule outlines a well-structured plan that will guide our project from inception to successful completion, ensuring that we meet our project goals and deliver a high-quality web application.
\bibliographystyle{unsrt}

\nocite{*}

\bibliography{prop}
\hrule
\vspace{.5in}

Guide \\
\vspace{.5in}

Coordinator
\end{document}
